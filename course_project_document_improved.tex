% !TeX program = pdflatex
\documentclass[12pt,a4paper]{article}

% Encoding and language support
\usepackage[utf8]{inputenc}
\usepackage[T2A]{fontenc}
\usepackage[bulgarian]{babel}

% Layout
\usepackage{geometry}
\geometry{margin=2.5cm}
\usepackage{microtype}

% Header and footer
\usepackage{fancyhdr}
\pagestyle{fancy}
\fancyhf{}
\fancyhead[L]{Каталог на видеоигри}
\fancyhead[R]{\leftmark}
\fancyfoot[C]{\thepage}
\setlength{\headheight}{15pt}

% Section formatting
\usepackage{titlesec}
\titleformat{\section}{\Large\bfseries}{\thesection.}{1em}{}
\titleformat{\subsection}{\large\bfseries}{\thesubsection}{1em}{}
\titleformat{\subsubsection}{\normalsize\bfseries}{\thesubsubsection}{1em}{}

% Spacing for paragraphs
\setlength{\parindent}{0pt}
\setlength{\parskip}{0.6em}

% Enumerations
\usepackage{enumitem}

% Hyperlinks
\usepackage[hidelinks]{hyperref}

% Rename contents
\renewcommand{\contentsname}{Съдържание}

\begin{document}

% Title page
\begin{titlepage}
  \vspace*{2cm}
  \begin{center}
    {\LARGE Софийски университет „Св. Кл. Охридски”\\[1ex]}
    {\large Факултет по математика и информатика\\[1ex]}
    {\large Бакалавърска програма „Софтуерно инженерство”}\\[4ex]
    {\large Предмет: XML технологии за семантичен Уеб}\\[1ex]
    {\large Зимен семестър, 2025/2026 год.}\\[6ex]
    {\Large Тема №0123: „Шаблон за оформление на документация на курсов проект“}\\[3ex]
    {\large Курсов проект}\\[3ex]
    \begin{tabular}{l}
      Автори: \\
      Ивайло~Кънчев, фак. номер~2MI0600305 \\
      Йоан~Байчев, фак. номер~0MI0600328 \\
    \end{tabular}
    \\[5ex]
    \vfill
    {\large януари, 2026~г.}\\[1ex]
    София
  \end{center}
\end{titlepage}

\pagenumbering{roman}
\tableofcontents
\clearpage
\pagenumbering{arabic}

\section{Въведение}
Настоящият документ представя разработката на курсов проект по „XML технологии за семантичен~Уеб“, чиято цел е създаване на каталог на видеоигри за развлечение. Проектът показва как чрез XML и свързаните с него технологии (XML Schema, XSLT, CSS и JavaScript) може да се моделира, валидира и визуализира богато текстово и графично съдържание за популярни игри – жанрове, компании (разработчици и издатели), платформи, дати на излизане и изображения. Основният проблем, който решаваме, е структурираното и консистентно представяне на многоизмерни данни за видеоигри, така че те да могат да се валидират автоматично и да се визуализират интерактивно без сървърна логика.

В контекста на предмета това демонстрира пълния технологичен цикъл: дефиниране на схема с ключове и референции, използване на непарснати XML ентитети за изображения, трансформация на съдържанието с XSLT директно в браузър и добавяне на динамика с JavaScript за търсене, филтриране и сортиране. Решението използва единичен XML документ с три основни колекции (жанрове, компании, игри), който се валидира срещу XML Schema с ключове и keyref зависимости (ID/IDREF и композитен ключ за компаниите). Изображенията са свързани чрез непарснати ентитети, за да се осигури ясна връзка между данните и графичните ресурси. Визуализацията е изцяло клиентска: XSLT оформя HTML таблица, CSS придава стил, а JavaScript реализира интерактивните функции.

Останалата част от документа е структурирана така: раздел \ref{sec:analysis} описва работния процес и структурата на съдържанието; раздел \ref{sec:design} разглежда дизайна и архитектурата на решението; раздел \ref{sec:ai} описва използването на изкуствен интелект в процеса; раздел \ref{sec:testing} представя тестването и резултатите; раздел \ref{sec:conclusion} обобщава заключенията и очертава възможности за бъдещо развитие; раздели \ref{sec:work-distribution}–\ref{sec:appendix} съдържат разпределението на работата, източниците и апендикси.

\section{Анализ на решението}
\label{sec:analysis}

\subsection{Работен процес}
Работният процес на „Каталог на видеоигри за развлечение“ обхваща целия цикъл от подготовката на данните, през валидацията и трансформацията, до интерактивната визуализация в браузър. Входният документ \texttt{catalog.xml} съдържа три колекции: жанрове (\texttt{genres}), компании (\texttt{companies}) и игри (\texttt{games}). Всеки жанр има уникален идентификатор и име; всяка компания — име, държава и роля; всяка игра реферира към жанр, към една или повече компании по роля, съдържа заглавие, платформи, резюме, дата на излизане и препратка към изображение чрез XML ентитет. Ентитетите за изображения (PNG файлове в директорията \texttt{images/}) са дефинирани в \texttt{DOCTYPE} като непарснати, за да се поддържа ясно разграничение между данни и медийни ресурси.

Този процес може да се разглежда като тройка „вход–обработка–изход“. \textbf{Входът} е един XML документ с данни + DOCTYPE ентитети + XSD. \textbf{Обработката} включва (1) валидация срещу XSD; (2) XSLT трансформация в браузъра; (3) зареждане на CSS и инициализация на JavaScript за интерактивност. \textbf{Изходът} е HTML страница с таблица и контроли за търсене/филтър/сортиране, която може да се отвори директно или през прост локален сървър. Тъй като няма бекенд, целият жизнен цикъл е файлов и преносим.

Първата стъпка е валидиране срещу XML Schema (\texttt{catalog.xsd}), която дефинира типове за идентификатори, ключове и референции, както и изброени стойности за атрибута \texttt{image}. Ключовете \texttt{genreKey} и \texttt{companyKey} (композитен ключ по \texttt{name}+\texttt{role}) и съответните ключови връзки (\texttt{keyref}) гарантират, че всяка игра сочи валиден жанр и валидни компании за разработка/издаване. След валидирането XML се трансформира чрез XSLT стил (\texttt{catalog.xsl}) директно в браузъра. Трансформацията създава HTML таблица с колони като заглавие, дата на излизане, жанр, платформи, разработчик, издател, описание и изображение; изображенията се вмъкват чрез функцията \texttt{unparsed-entity-uri(@image)}. JavaScript осигурява интерактивност: търсене, филтър и сортиране. CSS определя визуалния облик и удобството за ползване.

Изходът е една HTML страница, която може да се зареди директно в браузър без сървърна логика. Всички зависимости са файлови: XML, XSD, XSLT, CSS и изображения. Работният процес е лесно повторим: при добавяне на нова игра се обновява XML-ът и, ако има ново изображение, се актуализира \texttt{DOCTYPE}/XSD енума; валидирането и трансформацията се извършват отново без промени в кода.

\textbf{Примерен поток (конкретно)}: входът е \texttt{catalog.xml} + \texttt{catalog.xsd} + PNG в \texttt{images/}. Стъпки: (1) \texttt{xmllint} валидира; (2) браузър прилага \texttt{catalog.xsl}; (3) генерира се HTML с таблица и контроли; (4) JS активира търсене/филтри/сортиране; (5) потребителят преглежда/филтрира списъка. Очакван резултат: осем реда, всеки със снимка, коректни жанрове и компании.

\subsection{Структура на съдържанието}
Структурата на съдържанието е описана формално в \texttt{catalog.xsd}. Моделът включва:
\begin{itemize}[topsep=0pt]
  \item \textbf{Жанрове} \texttt{genres/genre} – атрибути \texttt{id} (тип ID) и \texttt{name}. Жанровете са седем: \emph{Action-adventure}, \emph{Action RPG}, \emph{First-person action}, \emph{Sandbox/Survival}, \emph{Puzzle}, \emph{Platformer}, \emph{Role-playing}. Ключ \texttt{genreKey} гарантира уникалност на идентификаторите, а игрите реферират към жанровете чрез \texttt{genreId}.
  \item \textbf{Компании} \texttt{companies/company} – атрибути \texttt{name}, \texttt{country} и \texttt{role}. Композитният ключ \texttt{companyKey} (\texttt{name}+\texttt{role}) позволява една компания да бъде едновременно разработчик и издател. В игрите елементът \texttt{companyRef} съдържа име и роля и се валидира чрез ключова връзка \texttt{companyRefKey}.
  \item \textbf{Игри} \texttt{games/game} – атрибути \texttt{id}, \texttt{genreId}, \texttt{image} (\texttt{ENTITY}, ограничен чрез \texttt{ImageEntityType}) и \texttt{releaseDate}. Поделементи \texttt{title}, \texttt{platforms}, \texttt{summary} и \texttt{companies/companyRef}. Структурата допуска множество разработчици и издатели, а датите са в ISO формат (YYYY-MM-DD) за улеснено сортиране.
  \item \textbf{Медийни ресурси} – в \texttt{DOCTYPE} са дефинирани непарснати ентитети (\texttt{totk}, \texttt{eldenring}, \texttt{doometernal}, \texttt{minecraft}, \texttt{tetris}, \texttt{witcher3}, \texttt{mario}, \texttt{bg3}), които сочат към PNG файлове. XSD енума \texttt{ImageEntityType} дублира тези стойности и гарантира съответствие.
\end{itemize}
Връзките между колекциите се реализират чрез \texttt{ID}/\texttt{IDREF} и \texttt{key}/\texttt{keyref}, което позволява автоматична проверка за консистентност. Форматът е лесно разширяем: добавяне на нови игри, жанрове или компании изисква само попълване на XML-а и, при ново изображение, обновяване на \texttt{DOCTYPE}/XSD.

\textbf{Обзор на игрите (кратко):}
\begin{itemize}[topsep=0pt]
  \item \texttt{game1} – The Legend of Zelda: TOTK, жанр Action-adventure, разработчици Nintendo EPD/Monolith Soft, издател Nintendo, дата 2023-05-12.
  \item \texttt{game2} – Elden Ring, жанр Action RPG, FromSoftware (dev/publisher), Bandai Namco (publisher), дата 2022-02-25.
  \item \texttt{game3} – DOOM Eternal, жанр First-person action, id Software / Bethesda, дата 2020-03-20.
  \item \texttt{game4} – Minecraft, жанр Sandbox/Survival, Mojang (dev/pub), дата 2011-11-18.
  \item \texttt{game5} – Tetris, жанр Puzzle, Alexey Pajitnov / Vadim Gerasimov / The Tetris Company, дата 1984-06-06.
  \item \texttt{game6} – The Witcher 3, жанр Action RPG, CD Projekt RED / CD Projekt, дата 2015-05-19.
  \item \texttt{game7} – Super Mario Odyssey, жанр Platformer, Nintendo EPD / 1-Up Studio / Nintendo, дата 2017-10-27.
  \item \texttt{game8} – Baldur's Gate 3, жанр Role-playing, Larian Studios (dev/pub), дата 2023-08-03.
\end{itemize}

\textbf{Процедура при добавяне на нова игра (поддръжка):}
\begin{enumerate}[leftmargin=2em]
  \item Добави PNG в \texttt{images/} и дефинирай непарснат ентитет в \texttt{DOCTYPE}.
  \item Обнови XSD \texttt{ImageEntityType} (\texttt{enumeration}) с новото име.
  \item Вмъкни запис в \texttt{games} с валиден \texttt{genreId} и \texttt{companyRef} стойности по роля.
  \item По желание добави нов жанр/компания в съответната колекция.
  \item Пусни \texttt{xmllint} за валидация и визуална проверка в браузър.
\end{enumerate}

\textbf{Разширяемост}: нова игра изисква добавяне на елемент в \texttt{games}, валиден \texttt{genreId} към \texttt{genres}, референции към компании и, при ново изображение, обновяване на \texttt{DOCTYPE} и XSD \texttt{enum}. Ако се въведат поджанрове, може да се добави \texttt{parentId} и нови \texttt{key}/\texttt{keyref}. За платформи може да се дефинира отделна колекция с \texttt{enum}, за да се избегнат свободни текстове.

\subsection{Тип и представяне на съдържанието}
Съдържанието комбинира текст и графика. Всички текстови елементи (заглавия, резюмета, списъци с платформи, имена на компании, дати) са в UTF‑8 и следват ISO форматите. Резюметата са чист текст без HTML тагове, за да се предотврати инжектиране на код. Графичните ресурси са осем PNG изображения с размер 1.5–3.2~MB, достъпни чрез непарснати ентитети; XSLT ги вмъква с \texttt{unparsed-entity-uri()}.

Фронтендът представлява HTML таблица с удобни контроли за търсене и филтриране. CSS добавя стил – светъл фон, цветен хедър, редуващи се редове, ограничена ширина на изображенията. JavaScript осигурява динамика без нужда от презареждане.

\section{Дизайн}
\label{sec:design}

\subsection{Архитектура и компоненти}
Проектът е реализиран като изцяло клиентско приложение, което не изисква сървър или база данни. Основните компоненти са:
\begin{enumerate}[leftmargin=2em]
  \item \textbf{Данни} – \texttt{catalog.xml} съдържа всички записи за жанрове, компании и игри. \texttt{DOCTYPE} определя връзките към изображенията.
  \item \textbf{Схема} – \texttt{catalog.xsd} дефинира типове, ключове и ключови връзки, като гарантира валидност на данните.
  \item \textbf{Трансформация} – \texttt{catalog.xsl} генерира HTML и включва JavaScript за интерактивност; \texttt{style.css} задава стил.
  \item \textbf{Резултат} – една HTML страница, която може да се отвори директно и включва търсачка, филтър по жанр и сортиране на колони.
\end{enumerate}

\subsection{XML ентитети}
Медийните ресурси са дефинирани като непарснати ентитети в \texttt{DOCTYPE}:
\begin{verbatim}
<!ENTITY totk        SYSTEM "images/totk.png"         NDATA png>
<!ENTITY eldenring   SYSTEM "images/eldenring.png"    NDATA png>
...
\end{verbatim}
Използването на ентитети позволява централизирано управление на изображенията. В XSLT се използва \texttt{unparsed-entity-uri(@image)} за превръщане на ентитета в URL за \texttt{<img>}, което улеснява поддръжката и валидирането чрез XSD енума \texttt{ImageEntityType}.

\subsection{Валидация и тестове}
Преди трансформацията данните се валидират с помощта на команда:
\begin{verbatim}
xmllint --noout --schema catalog.xsd catalog.xml
\end{verbatim}
Това потвърждава правилно изпълнени \texttt{ID}/\texttt{IDREF} връзки, композитни ключове и типове. След трансформацията се извършва визуална проверка в няколко браузъра (Chrome, Firefox, Edge), за да се гарантира коректно зареждане на изображенията и работещи функции за търсене, филтър и сортиране.

\subsection{Поток и роля на компонентите}
\begin{enumerate}[leftmargin=2em]
  \item Зареждане на \texttt{catalog.xml} в браузър, при което се прилага \texttt{catalog.xsl}.
  \item XSLT генерира HTML таблица и вгражда JavaScript; \texttt{unparsed-entity-uri()} резолвира \texttt{@image} към PNG.
  \item CSS придава стил; JS активира филтър по жанр, търсене и сортиране (стрингово и по дати).
  \item При промяна на данни: обновяване на XML (+ \texttt{DOCTYPE}/XSD \texttt{enum} при ново изображение) и повторно валидиране; трансформацията остава непроменена.
\end{enumerate}

\subsection{Разширяемост и алтернативи}
\textbf{Разширения}: поджанрове и PEGI/ESRB рейтинги; филтри по платформи и диапазон дати; PDF изход чрез XSL‑FO; локализация на интерфейса; по‑строга типизация на платформи чрез отделен \texttt{enum}/колекция; интеграция с външни API. \textbf{Алтернативи}: JSON/HTML без XSLT (по‑малко демонстрация на XML трансформации); бекенд/БД (по‑тежка инфраструктура); външни шрифтове/фреймуърци (добавят зависимости).

\subsection{Обобщение}
Дизайнът демонстрира цялостна XML-първа архитектура: дефиниране и валидиране на структурата, използване на непарснати ентитети за графика, трансформация с XSLT и клиентска интерактивност с JavaScript, оформена с CSS. Връзките между жанрове, компании и игри са строго типизирани чрез \texttt{key}/\texttt{keyref} и \texttt{ID}/\texttt{IDREF}, а изображенията са управлявани през \texttt{ENTITY} декларации.

\section{Използване на изкуствен интелект}
\label{sec:ai}
Изкуственият интелект бе използван като асистент при планиране, генериране на чернови и проверка на формулировки. Модел от типа LLM (GPT‑семейство) беше подканван с ясни ограничения: жанрове, роли developer/publisher, ISO дати, UTF‑8, непарснати \texttt{ENTITY} имена за изображения, както и указания за структурата на XML/XSD/XSLT. Генерираните текстове (резюмета, описания, документация) бяха преглеждани и редактирани ръчно; кодът (XML/XSD/XSLT/CSS/JS) беше проверен за синтактична коректност и съвместимост. За изображения се използваха стилизирани/генерирани варианти, за да се избегнат авторски права. SWOT: Strengths – бързо съдържание и консистентна структура; Weaknesses – нужда от ръчна проверка; Opportunities – разширяване с рейтинги, интеграции, локализация; Threats – остаряване на данни и варираща XSLT поддръжка.

\section{Тестване}
\label{sec:testing}
Тестването включваше:
\begin{itemize}
  \item \textbf{Валидация на XML} – проверка с \texttt{xmllint} за валидност спрямо XSD.
  \item \textbf{Визуален преглед} – зареждане на \texttt{catalog.xml} и статичния \texttt{test\_output.html} в различни браузъри.
  \item \textbf{Функционални тестове} – тестване на търсачката, филтъра по жанр и сортирането на всички колони, проверка за правилно зареждане на изображенията.
\end{itemize}
Среда: Chrome/Edge/Firefox (актуални версии), локален прост HTTP сървър при нужда. Сценарии: сортиране по дата (ISO), търсене с главни/малки букви, филтър по жанр, проверка на счупен ентитет (очаквана грешка при валидация), липсващ жанр (очакван отказ от XSD). Препоръчителна автоматизация: \texttt{make validate} с \texttt{xmllint --noout --schema catalog.xsd catalog.xml} и \texttt{xsltproc catalog.xsl catalog.xml > /tmp/out.html}.
Допълнително бяха проверени: коректно парсиране на ISO дати при сортиране; устойчивост на търсачката към главни/малки букви; липса на счупени \texttt{src} за изображения; отсъствие на грешки в конзолата. Препоръчително е при всяко добавяне на игра да се пусне \texttt{xmllint} и визуален преглед в поне два браузъра.

\section{Заключение и възможно бъдещо развитие}
\label{sec:conclusion}
Проектът демонстрира как чрез XML и свързани технологии може да се реализира богат, интерактивен каталог без нужда от бекенд. Предимствата са ясната структура, автоматичната валидация, лесната поддръжка и преносимост. Основните ограничения са ръчната актуализация и зависимостта от XSLT поддръжка в браузърите.

В бъдеще каталогът може да се разшири с оценки и ревюта, филтри по платформи и времеви интервали, локализация на интерфейса, генериране на PDF чрез XSL‑FO и по‑строга типизация (например отделна колекция за платформи). Други идеи включват йерархия на жанровете, интеграция с външни API за автоматично обновяване на данните и версия на мобилно приложение. Планирани разширения: поддръжка на поджанрове и PEGI/ESRB рейтинги; филтри по платформа и диапазон дати; PDF изход чрез XSL‑FO; локализация на интерфейса; по‑строга типизация на платформи чрез отделен \texttt{enum}/колекция; интеграция с външни API за автоматично обновяване на каталога.
Практическа приложимост: подходът е подходящ за малки/средни каталози, офлайн справочници и учебни демонстрации. Поддръжка: при обновяване се следва checklist от апендикса (данни → \texttt{DOCTYPE}/XSD → \texttt{xmllint} → визуален преглед → по желание \texttt{xsltproc}). При нужда от по-мащабно решение може да се въведе бекенд или API синхронизация, но текущият файл-базиран модел остава лек и преносим.

\section{Разпределение на работата}
\label{sec:work-distribution}
Работата по проекта беше разделена така:
\begin{itemize}
  \item \textbf{Събиране и проверка на данни за игрите, жанровете, платформите и компаниите}: Автор A, Автор B.
  \item \textbf{Проектиране на XML структурата и XSD схемата}: Автор A.
  \item \textbf{Разработка на XSLT, CSS и JavaScript за визуализация и интерактивност}: Автор B.
  \item \textbf{Подготовка на изображения и \texttt{DOCTYPE} ентитети}: Автор A, Автор B.
  \item \textbf{Документиране и преглед на код}: Автор A, Автор B.
\end{itemize}

\section{Източници и литература}
Използвани бяха официални сайтове на компаниите (Nintendo, FromSoftware, Bethesda, Mojang, Larian и др.), енциклопедични източници за проверка на дати и платформи (например Wikipedia), спецификациите на W3C за XML, XML Schema и XSLT, документацията на \texttt{libxml2}/\texttt{xmllint}, както и курсови материали по „XML технологии за семантичен Уеб“. Основни референтни линкове: \url{https://www.w3.org/TR/xml/}, \url{https://www.w3.org/TR/xmlschema-1/}, \url{https://www.w3.org/TR/xslt}. Други източници могат да бъдат добавени при нужда.

\section{Апендикс}
\label{sec:appendix}
\begin{itemize}
  \item \textbf{Файлова структура} – \texttt{catalog.xml}, \texttt{catalog.xsd}, \texttt{catalog.xsl}, \texttt{style.css}, \texttt{test\_output.html}, \texttt{README.md}, \texttt{course\_project\_report.md} и директория \texttt{images/} с осем PNG.
  \item \textbf{Валидация}: \texttt{xmllint --noout --schema catalog.xsd catalog.xml}.
  \item \textbf{Трансформация извън браузър}: \texttt{xsltproc catalog.xsl catalog.xml > output.html}.
  \item \textbf{Роля на файловете}: \texttt{catalog.xml} (данни), \texttt{catalog.xsd} (схема и ограничения), \texttt{catalog.xsl} (трансформация + JS), \texttt{style.css} (визуален стил), \texttt{images/} (медии), \texttt{test\_output.html} (референтен резултат), \texttt{course\_project\_report.md}/\texttt{course\_project\_document\_improved.tex} (документация).
  \item \textbf{Workflow checklist}: добави/актуализирай данни в XML; обнови \texttt{DOCTYPE}/XSD enum при ново изображение; пусни \texttt{xmllint}; провери визуално в браузър; при нужда генерирай статичен HTML с \texttt{xsltproc}.
  \item \textbf{Допълнителни материали}: екранни снимки, инструкции за локално отваряне (\texttt{python -m http.server 8000} и \url{http://localhost:8000/catalog.xml}) и др.
\end{itemize}

\end{document}
